\begin{titlepage}
\begin{center}

% Upper part of the page. The '~' is needed because \\
% only works if a paragraph has started.
\includegraphics[width=5in]{../includes/Robots-color}

\vspace{1cm}

\textsc{\Large Slower Than Dirt Tunebook}\\[0.5cm]


\begin{minipage}{6in}
\setlength{\parindent}{15pt} % Default is 15pt.
\setlength{\parskip}{1em plus.4em minus.3em}

This tunebook contains a core repertoire of tunes to be played at the
\href{http://slowerthandirt.org/}{Slower Than Dirt} slow/beginner
old-time jam in Seattle. Tunes from this book will be played at every
Slower Than Dirt jam; which ones will be announced in advance each
month on the related mailing list and web site.

Tools used in the preparation of this tunebook include abcm2ps,
EasyABC and {\LaTeX}. The idea came from Paul Hardy's tunebooks at
\url{http://www.pghardy.net/concertina/tunebooks/}, and assistance
with ABC formatting came from Pete Showman and his
\href{http://www.showman.org/jams/oti/}{South Bay Old-Time Jam}
transcriptions.  There is a GitHub repository of files at
\url{https://github.com/rjl20/abc-tunebook} containing the source code
for this book.

\end{minipage}

\vfill

% Bottom of the page

\includegraphics[width=1in]{../includes/cc-by-sa}\\
This work is licensed under the Creative Commons
Attribution-ShareAlike 4.0 International License.\\
To view a copy of this license, visit 
\url{http://creativecommons.org/licenses/by-sa/4.0/}
\end{center}
\end{titlepage}

\clearpage


\section*{A note about copyright and licensing}

Unless otherwise noted, all transcriptions and text in this book are
licensed under a Creative Commons Attribution-ShareAlike 4.0
International License. This means that the transcriptions are, to the
extent possible by law, freely available for you to use however you
want, as long as you give us attribution (a link back to
\url{http://slowerthandirt.org/} would be ideal, if possible) for the bits
you use and as long as whatever work you incorporate ours into is also
licensed under the same terms. For more details, visit
\url{https://creativecommons.org/licenses/by-sa/4.0/}

In practice, thanks to the murky state of copyright as it applies to
music, this doesn't mean an awful lot. \textbf{The license applies to the
transcriptions only, not to the underlying compositions.} If you want
to do something with sheet music you find here, you're still on your
own for making sure it's legal for you to do it. Some tunes we think
are in the public domain might not be. If it turns out that we didn't
get permission from a composer or publisher to include a transcription
of their tune here, our transcription being CC-licensed doesn't mean
that you can print our transcription in your book. It just means that
we won't be the ones coming after you for royalties and/or
penalties. Even if we do get permission from a composer to publish one
of their tunes here, that doesn't necessarily mean you can record it
on an album or perform it in public without paying them a royalty.

If you'd like to use transcriptions or text from Slower Than Dirt but
are unable to comply with the terms of the Creative Commons
Attribution-ShareAlike 4.0 International License, please use the
contact form at \url{http://slowerthandirt.org/contact/} to let us know what
you'd like to do.

If you are a composer or musician whose intellectual property is being
used here without permission, please let us know by emailing
{\tt copyright@slowerthandirt.org} — we'd really like to know when something
that we think of as ``traditional'' is actually a recent composition, so
we can get permission and give appropriate credit or remove the tune
as necessary.

Neither of the founders of Slower Than Dirt is a lawyer. None of this
is legal advice. For a canned rant about how the Copyright Act of 1976
outlawed folk music, talk to Josh using either of the contact methods
above.

\clearpage

\section*{Why a tune book?}

Some people will tell you that sheet music has no place in
old-time. These people are wrong. I can think of a few tunes popular
in jam sessions and at square dances right now which are almost
certainly out in the world because someone found them in an old
manuscript, tried them out, and liked them enough to keep playing
them. With no recordings and nobody passing the tune along through
direct transmission, these tunes would be lost if they hadn't been
written down and then read and reinterpreted later.

That's not what this book is for. But it's not axiomatic that old-time
and sheet music are at odds.

Some people will tell you that you can't learn a tune from sheet
music, because the essence of a tune is in the nuances of the
performance--the subtle details and variations that sheet music can't
capture. They're not wrong, exactly, because it's true that it's very
difficult to capture much of the nuance of a performance in standard
notation. But I think they misunderstand what sheet music is for, what
a musician's relationship to sheet music is. A musician is not a
machine for turning sheet music into sound. We have actual machines
for that. When I'm reading a new-to-me tune from sheet music, I'm
bringing my understanding of what this kind of music usually sounds
like, or how I like it to sound, with me. I am \emph{interpreting} the
sheet music, not \emph{executing} it. I'm probably not going to play
the tune the way you learned it from Uncle ``Vern'' Hotchkiss, but
this isn't an imitation contest; that's ok.

On the other hand, there is definitely a danger to thinking that any
particular book's version of a tune is authoritatively ``how the tune
goes'', or in always trying to play it exactly how it appears, note
for note. Once you've learned a tune, sheet music can be a useful
reminder of how the tune goes if you get lost, but learning a tune
involves making it your own. Play it how it sounds in your head, not
how it looks on the page. (If you're at a jam, make sure to pay
attention to whether how it sounds in your head works with how the
tune leader is playing it.)  If you're reading it off the page every
time, think about whether you're actually learning and playing the
tune, or if you're making yourself into a machine for turning sheet
music into sound. There's nothing wrong with sight-reading, but part
of the point of this jam is for people to learn and practice a new way
of learning.

So what is this book for? Learning music by ear is a different skill
than learning from sheet music. And most jams, including Slower Than
Dirt, aren't really teaching sessions, where a tune gets broken into
parts and taught phrase by phrase. When you're a beginning musician
and haven't yet picked up the skill of learning a tune on the fly,
going to a jam where you don't know any of the tunes can be
frustrating. So this book contains a core repertoire from which some
number of tunes will be picked each month and definitely played at the
Slower Than Dirt jam. If you want, you can practice some of them at
home in advance and know that you'll be able to play along.

Some of the transcriptions in this book attempt to capture the essence
of an individual performance, while some of them are stripped down to
what I consider the bones of the tune. We may not play a tune exactly
as it's written here at the jam, so you can think of this book as more
a cheat sheet than Holy Writ. How I lead a tune at the jam will at
least resemble the version here, though, so at least you won't run
into the problem of ``oh, this is the other tune with that name''.


\section*{On learning by ear}

I don't know how other people do it, but I can tell you what worked
for me. I listened to a ton of old-time music. Constantly. On the bus,
walking around, at home while catching up on facebook: all the
time. For months. If I wasn't doing something else that required
attention, I was probably listening to old-time music on headphones. I
wasn't necessarily paying a lot of attention to the music, but it was
a constant presence.

The first few months got me used to some of the musical conventions of
old-time. I started being able to identify phrases or riffs I'd heard
before in other tunes. I started being able to predict what general
shape a tune might have. What those first months of listening got me
was a basic understanding that I didn't get by growing up with that
music all around me in my community. It got me familiarity.

Eventually, instead of listening to the entire collection on shuffle,
I picked a dozen tunes I liked and put them in a playlist, and
listened to nothing but them for a month. I must have listened to
those dozen tunes a hundred times. Each. But by the end of that month,
I could hear those tunes in my head. I could hum along. I hadn't set
out to learn the tunes, at least not in a way I understood as
intentional learning. But I knew those tunes. Not that I could play
them on an instrument, but humming them was the important part.

Once you can hum a tune, you can compare what's happening with your
instrument when you play it against what's happening in your head. It
can be frustrating to know how a thing ought to work but not to be
able to make it happen, but that's a mechanical issue with your
instrument and gets better with practice. Once I could hum the tunes
from my playlist, and actually kind of play a couple of them on the
fiddle, I picked another dozen tunes and made a new playlist and
repeated the process.

I did this four or five times, listening only to a dozen or so tunes
at a time until they were in my head enough that I could hum or
whistle them. And patterns started to emerge. There were phrases that
appeared across tunes, little ornaments that could be stacked in
different ways to produce different effects. I started hearing the
pieces the tunes were built from. It was like slowly discovering the
shapes of a lego set. I could start swapping different blocks in and
seeing how they worked. Meanwhile, I was practicing playing tunes from
earlier in the cycle, building the actual fiddling skill.

I don't think I'm a particularly good fiddler still, but I think I may
now be fairly decent at picking up a new tune, as long as it's in an
idiom I've been immersing myself in. I'm not going to pick up a Cajun
tune as quickly, or a Brazilian choro piece. For Appalachian and
midwestern old-time, though, I can identify the building blocks of a
tune I haven't heard before and assemble them in my head, rather than
trying to take in a whole tune all at once or note by note. What I'm
getting at is that, for me at least, ``playing the fiddle'' and
``learning new tunes'' are almost completely separate skills. I can't
say that what's worked for me will work for you, but I do think that
the critical piece is being able to hear a tune in your head before
trying to play it on an instrument. Whatever gets you there, do
that. And do it a lot, because just like learning the physical skill
of playing an instrument, it's something that takes practice.

\section*{About ``potatoes''\footnote{For the story of why they're called ``potatoes'', see \url{https://tinyurl.com/why-potatoes}} and ``boom-chuck''}

Often a fiddler will start a tune with ``potatoes'', or four beats of
introductory shuffling. It usually sounds like (has the rhythm of)
``one and-a two and-a three and-a four'', leaving space at the end for
a pick-up into the tune. This should give you a sense of three
things: what key the tune is in, what speed it's going to be played
at, and its rhythm. Listen for what notes are being played -- if it's
just one note, it's probably the root note of the key the tune is
in. If it's two notes, it'll probably be the root and fifth (D and A
for the key of D) or root and third (D and F$\sharp$) of the key.

The tempo is set by the longer notes -- the ``one'', not the
``and-a''.  Each of those corresponds to one beat, and which note
length corresponds to a beat is dependent on the time signature of the
tune. Most of the tunes in this book are in ``cut time'', or
$\tfrac{2}{2}$, which means that there are two half notes per measure,
each of which is one beat.\footnote{You might think that
$\tfrac{2}{2}$ is the same as $\tfrac{4}{4}$, because there are the
same number of quarter notes in a measure of both, but in
$\tfrac{2}{2}$ there are two beats -- where you would tap your foot
-- per measure, and in $\tfrac{4}{4}$ there are four.} If you're
playing guitar backup in a ``boom-chuck'' style, each beat gets a boom
and a chuck, with the boom (the bass note) on the beat and the chuck
(the strum) in between.

Rhythm is a little trickier, but listen to how regular the notes
are. Is the ``and'' the same length as the ``a'', or is it longer? If
it's longer, the tune is probably going to feel a little bouncier, and
if it's the same length it's probably going to feel a bit more
driving. If there's no ``and-a'', then they might be starting off a
jig (of which there are none currently in this book) or something else
which doesn't sound right with the usual shuffle intro.

If you're starting off a tune, keep those in mind when playing the
``potatoes''. If you aren't sure what key a tune is in, the chord it
ends on is usually (but not always) a safe bet. Try to feel the speed
of the tune as you want to play it before you start playing; maybe tap
your foot for a bit and hum the tune to yourself to make sure it's
where you want it to be. Then you can start the ``potatoes''. Keep
tapping your foot, though, and make sure you're starting the tune
where it wants to start -- if there's a pick-up, start it before the
``one'' of the first measure, not right on the ``one'', and if there's
no pick-up, start right on the ``one'', not where the final ``and''
would go.

That's a lot of words which probably over-complicate things, so to 
summarize: listen for the beat.



\cleardoublepage
